\documentclass[]{elsarticle} %review=doublespace preprint=single 5p=2 column
%%% Begin My package additions %%%%%%%%%%%%%%%%%%%
\usepackage[hyphens]{url}

  \journal{Journal of Transport \& Health} % Sets Journal name


\usepackage{lineno} % add
\providecommand{\tightlist}{%
  \setlength{\itemsep}{0pt}\setlength{\parskip}{0pt}}

\usepackage{graphicx}
\usepackage{booktabs} % book-quality tables
%%%%%%%%%%%%%%%% end my additions to header

\usepackage[T1]{fontenc}
\usepackage{lmodern}
\usepackage{amssymb,amsmath}
\usepackage{ifxetex,ifluatex}
\usepackage{fixltx2e} % provides \textsubscript
% use upquote if available, for straight quotes in verbatim environments
\IfFileExists{upquote.sty}{\usepackage{upquote}}{}
\ifnum 0\ifxetex 1\fi\ifluatex 1\fi=0 % if pdftex
  \usepackage[utf8]{inputenc}
\else % if luatex or xelatex
  \usepackage{fontspec}
  \ifxetex
    \usepackage{xltxtra,xunicode}
  \fi
  \defaultfontfeatures{Mapping=tex-text,Scale=MatchLowercase}
  \newcommand{\euro}{€}
\fi
% use microtype if available
\IfFileExists{microtype.sty}{\usepackage{microtype}}{}
\bibliographystyle{elsarticle-harv}
\ifxetex
  \usepackage[setpagesize=false, % page size defined by xetex
              unicode=false, % unicode breaks when used with xetex
              xetex]{hyperref}
\else
  \usepackage[unicode=true]{hyperref}
\fi
\hypersetup{breaklinks=true,
            bookmarks=true,
            pdfauthor={},
            pdftitle={Using environmental audits and photo-journeys to compare objective attributes and cyclists' perceptions along cycling routes},
            colorlinks=false,
            urlcolor=blue,
            linkcolor=magenta,
            pdfborder={0 0 0}}
\urlstyle{same}  % don't use monospace font for urls

\setcounter{secnumdepth}{0}
% Pandoc toggle for numbering sections (defaults to be off)
\setcounter{secnumdepth}{0}

% Pandoc citation processing

% Pandoc header



\begin{document}
\begin{frontmatter}

  \title{Using environmental audits and photo-journeys to compare objective
attributes and cyclists' perceptions along cycling routes}
    \author[School of Geography and Earth Sciences]{Elise Desjardins\corref{Corresponding Author}}
   \ead{desjae@mcmaster.ca} 
    \author[Department of Human Geography]{Christopher D. Higgins}
   \ead{cd.higgins@utoronto.ca} 
    \author[School of Geography and Earth Sciences]{Darren M. Scott}
   \ead{scottdm@mcmaster.ca} 
    \author[Department of Health Research Methods Evidence and Impact]{Emma Apatu}
   \ead{apatue@mcmaster.ca} 
    \author[School of Geography and Earth Sciences]{Antonio Páez}
   \ead{paezha@mcmaster.ca} 
      \address[School of Geography and Earth Sciences]{School of Geography and Earth Sciences, McMaster University, 1280 Main
Street West, Hamilton, Ontario L8S 4K1}
    \address[Department of Human Geography]{Department of Human Geography, University of Toronto - Scarborough, 1265
Military Trail, Toronto, Ontario M1C 1A4}
    \address[Department of Health Research Methods Evidence and Impact]{Department of Health Research Methods, Evidence, and Impact, McMaster
University, 1280 Main Street West, Hamilton, Ontario L8S 4K1}
      \cortext[1]{Corresponding Author}
  
  \begin{abstract}
  \textbf{\emph{Background}}: Cycling is known to have many health
  benefits. For this reason, transport planners and public health
  officials in Canada increasingly aim to encourage cycling for transport.
  On- and off-street infrastructure is often implemented to facilitate
  cycling and planners rely on a range of tools for informing the design
  of the network of facilities. This mixed methods study compares
  objectively measured attributes and cyclists' perceptions of the built
  environment along inferred cycling routes in Hamilton, Ontario.\\
  \textbf{\emph{Methods}}: Environmental audits were conducted along six
  cycling routes in Hamilton to document the attributes that might support
  or hinder cycling. The routes were inferred based on the output of a
  model of cycling flows. Cyclists, 9 male and 5 female, then participated
  in semi-structured interviews where a form of photo elicitation, which
  we call photo-journeys, was used to explore their perceptions and
  preferences. Interview data were analyzed using both inductive and
  deductive thematic analysis based on the categories of the audit
  instrument.\\
  \textbf{\emph{Results}}: Cyclists prefer routes that have dedicated
  cycling infrastructure, or residential streets with low volumes of
  traffic even if they lack infrastructure. They dislike routes with busy
  arterial roads or that lack cycling infrastructure. Their experience and
  knowledge of cycling in a city transitioning to be more bicycle-friendly
  revealed preferences that can help to improve existing infrastructure
  and cycling routes, which may also help to reduce barriers for
  non-cyclists.\\
  \textbf{\emph{Conclusions}}: Photo-journeys are an innovative and
  practical approach to explore perceptions of regular cyclists, which can
  be leveraged to inform policies and interventions to make cycling routes
  and infrastructure safer and more attractive. Transport planners in
  developing cycling cities should pay attention to both the objective
  attributes of the built environment and how it is perceived by the
  public.
  \end{abstract}
  
 \end{frontmatter}




\end{document}


